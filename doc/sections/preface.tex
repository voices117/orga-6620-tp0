\chapter*{Pr\'ologo\markboth{Preface}{Preface}}\label{ch:preface}
\addcontentsline{toc}{chapter}{Pr\'ologo}

\section{Objetivos}

Familiarizarse con las herramientas de software que usaremos en los siguientes trabajos,
implementando un programa y su correspondiente documentación que resuelvan el problema
descripto más abajo.

\section{Alcance}

Este trabajo práctico es de elaboración grupal, evaluación individual, y de carácter obligatorio para todos alumnos del curso.

\section{Requisitos}
El trabajo deber\'a ser entregado personalmente, en la fecha estipulada, con una car\'atula que contenga los datos completos de todos los integrantes.\\

Adem\'as, es necesario que el trabajo pr\'actico incluya, la presentaci\'on de los resultados obtenidos, explicando, cuando corresponda, con fundamentos reales, las causas o razones de cada resultado obtenido.\\

El informe deber\'a respetar el modelo de referencia que se encuentra en el grupo, y se valorar\'an aquellos escritos usando la herramienta TEX / LATEX.

\section{Recursos}

Usaremos el programa GXemul para simular el entorno de desarrollo que utilizaremos en este y otros trabajos prácticos, una máquina MIPS corriendo una versión reciente del sistema operativo NetBSD.
Durante la primera clase del curso presentaremos brevemente los pasos necesarios para la
instalación y configuración del entorno de desarrollo.
\\
\section{Fecha de entrega}

La última fecha de entrega y presentación ser\'a el jueves 5 de abril de 2018.
\\
\section{Informe}\label{informe}
El informe deberá incluir:
\begin{itemize}
\item Documentación relevante al diseño e implementación del programa.
\item Documentación relevante al proceso de compilación: cómo obtener el ejecutable a partir de los archivos fuente.
\item Las corridas de prueba, con los comentarios pertinentes.
\item El código fuente, en lenguaje C.
\item Este enunciado.
\end{itemize}
