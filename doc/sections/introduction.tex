\chapter{Introducci\'on}\label{ch:introduction}

\section{El comando wc}

El comando de Unix wc toma como entrada un archivo o stdin, y cuenta las palabras, las l\'ineas y la cantidad de caracteres que contiene.

\section{ Programas a desarrollar}
El programa a escribir, en lenguaje C, recibir\'a un nombre de archivo que
contiene texto (o el archivo mismo por stdin) e imprimir\'a por stdout la
cantidad de l\'ineas, palabras y caracteres que contiene, junto con el nombre
del archivo.

\section{Ejemplos}
Primero, usamos la opci\'on -h para ver el mensaje de ayuda:\

\begin{lstlisting}[numbers=left, tabsize=2, basicstyle=\fontsize{11}{13}\ttfamily, frame=single, caption={mensaje de ayuda}]
$ tp0 -h
Usage:
tp0 -h
tp0 -V
tp0 [options] file
Options:
-V, --version Print version and quit.
-h, --help Print this information.
-l, --words Print number of lines in file.
-w, --words Print number of words in file.
-c, --words Print number of characters in file.
-i, --input Path to input file.
Examples:
tp0 -w -i input.txt
Luego, lo usamos con un peque\~no fragmento de texto:
$echo -n "El tractorcito rojo que silbo y bufo" > entrada.txt
$tp0 -w -i entrada.txt
7 entrada.txt
$
\end{lstlisting}

\section{Mediciones}

Se deber\'a medir el tiempo insumido por el programa para el caso de los archivos alice.txt, beowulf.txt, cyclopedia.txt y elquijote.txt. \\

Graficar el tiempo insumido contra el tama\~no de muestra. Se deber\'a comprobar que el programa acepta las opciones dadas, y que reporta un error ante situaciones an\'omalas (como no encontrar el archivo solicitado). La ejecuci\'on del programa debe realizarse bajo el entorno MIPS.
